Para realizar la instalación se recomenda crear una carpeta general para el proyecto, dentro de esta carpetas separadas por cada lógica.

\subsection{Base de datos}
Para realizar la instalación de la base de datos se cuenta con las siguientes dos opciones:
\begin{itemize}
    \item Instalación local la cual permite realizar cambios que no afectan la lógica de desarrollo ni la de producción.
    \item Instalación producción la cual permite actualizar la base de datos que se comunica con todos los clientes del aplicativo.
\end{itemize}

Esta instalación es la que debe usarse para las diferentes etapas del desarollo de cualquier nueva funcionalidad.

Para realizarse se deben seguir los siguientes procesos:

\subsubsection*{Instalación MySQL}
\begin{enumerate}
	\item Instalar MySQL Server.
	\item Crear la base de datos.

	\begin{figure}
	\begin{lstlisting}[language=sql]
CREATE DATABASE chibchaweb;
        \end{lstlisting}
        \caption{Crear base de datos.}
        \label{fig:crear-bd}
        \end{figure}
    \item Crear un nuevo usuario.

    \begin{figure}
    \begin{lstlisting}[language=sql]
CREATE USER 'chibcha'@'localhost' identified by 'chibcha';
        \end{lstlisting}
        \caption{Crear usuario en base de datos.}
        \label{fig:crear-usuario}
        \end{figure}
	\item Otorgar y actualizar privilegios al usuario.

	\begin{figure}
	\begin{lstlisting}[language=sql]
GRANT ALL PRIVILEGES ON chibcha.* TO 'chibcha'@'localhost';
FLUSH PRIVILEGES;
        \end{lstlisting}
        \caption{Dar privilegios al usuario.}
        \label{fig:dar-privilegios}
        \end{figure}
\end{enumerate}

\subsubsection*{Configuración DBeaver}
Ya instalado el Dbeaver se procede a crear nueva conexión en DBeaver:
\begin{enumerate}
\item Abrir DBeaver.
\item Clic en "New Database Connection" (ícono de plug).
\item Eligir MySQL.
\item Llena los campos:
    \begin{itemize}
    \item Host:	localhost
    \item Port:	3306
    \item Database:	chibchaweb
    \item Username:	chibcha
    \item Password:	chibcha
    \end{itemize}
\item Clic en Test Connection.
    \begin{itemize}
	\item Permitir instalar el driver JDBC, solo aceptar y dejar que DBeaver lo descargue.
    \end{itemize}
\item Clic en Finish.
\end{enumerate}

\subsubsection*{Configuración Base de datos}
\begin{enumerate}
	\item Clonar el repositorio:

	\begin{figure}
	\begin{lstlisting}[language=bash]
git clone https://github.com/Rocnarx/Chibchaweb-Hosting-Platform-Database.git
    \end{lstlisting}
    \caption{Clonar repo Database.}
    \label{fig:clonar-db}
    \end{figure}
     \item Ejecutar en Dbeaver el archivo CreacionDB.sql
     \item Ejecutar en DBeaver el archivo DatosBasicos.sql

\end{enumerate}

\subsection{Backend}
\subsubsection*{Configurar requirimientos}
\begin{enumerate}
	\item Descargar e instalar Python 3.12
	\begin{itemize}
	\item Windows:
	Descargar de https://www.python.org/downloads/release/python-3120/
	\item Linux:

	\begin{figure}
	\begin{lstlisting}[language=bash]
sudo apt update
sudo apt install software-properties-common
sudo add-apt-repository ppa:deadsnakes/ppa
sudo apt update
sudo apt install python3.12 python3.12-venv python3.12-dev
        \end{lstlisting}
        \caption{Instalar Python 3.12}
        \label{fig:install-python}
        \end{figure}
	\end{itemize}
\end{enumerate}

\subsubsection*{Configurar Back}
\begin{enumerate}
    \item En la carpeta back crear un entorno virtual

    \begin{figure}
    \begin{lstlisting}[language=bash]
    python3 -m venv ChibchaBack
       \end{lstlisting}
       \caption{Crear venv.}
       \label{fig:crear-venv}
       \end{figure}
    \item Iniciar el entorno virtual


    \begin{figure}
    \begin{lstlisting}[language=bash]
    # -- Windows
    ./ChibchaBack/Scripts/activate
    #Linux
    source ChibchaBack/bin/activate
    \end{lstlisting}
    \caption{Activar venv}
    \label{fig:Activar-venv}
    \end{figure}

	\item Clonar el repositorio:

	\begin{figure}
	\begin{lstlisting}[language=bash]
    git clone https://github.com/Rocnarx/Chibchaweb-Hosting-Platform-Backend.git
    \end{lstlisting}
    \caption{Clonar repo backend.}
    \label{fig:Clonar-repo-backend}
    \end{figure}

    \item Instalar los paquetes requeridos que se encuentran en archivo requeriments.text:


    \begin{figure}
    \begin{lstlisting}[language=bash]
    pip install -r Chibchaweb-Hosting-Platform-Backend/requirements.txt
    \end{lstlisting}
    \caption{Instalar requerimientos back.}
    \label{fig:install-req}
    \end{figure}

    \item Crear archivo .env el cual nos permetira conectar el back con los demás componentes

    \item Configurar archivo .env con el siguiente texto

    \begin{figure}
    \begin{lstlisting}[language=bash]
FERNET_KEY= Solicitar al equipo de desarrollo
API_KEY= Solicitar al equipo de desarrollo
DATABASE_URL= mysql+pymysql://root:.. Solicitar al equipo de desarrollo
OPENAI_API_KEY= Solicitar al equipo de desarrollo
\end{lstlisting}
\caption{Editar .env back}
\label{fig:env-back}
\end{figure}

\end{enumerate}

\subsection{Front}

\subsubsection*{Configurar requerimientos}
\begin{enumerate}
    \item Instalar Node.js y npm
    \begin{itemize}
        \item Windows: Descargar e instalar desde nodejs.org (recomiendo la versión LTS)
        \item Linux:

        \begin{figure}[H]
        \begin{lstlisting}[language=bash]
    sudo apt update
    sudo apt install nodejs npm
        \end{lstlisting}
        \caption{Instalar node.}
        \label{fig:Instalar-node}
        \end{figure}
    \end{itemize}
\end{enumerate}

\subsubsection*{Configurar Front}

\begin{enumerate}
    \item Clonar el repositorio en la carpeta destinada a Front

    \begin{figure}
    \begin{lstlisting}[language=bash]
git clone https://github.com/Rocnarx/Chibchaweb-Hosting-Platform-Frontend.git
    \end{lstlisting}
    \caption{Clonar repo Front.}
    \label{fig:Clonar-repo-Front}
    \end{figure}
     \item Acceder a la carpeta de proyecto e instalar las dependencias

     \begin{figure}[H]
     \begin{lstlisting}[language=bash]
     cd Chibchaweb-Hosting-Platform-Frontend
     npm install
     \end{lstlisting}
     \caption{Instalar dependencias front.}
     \label{fig:req-front}
     \end{figure}
     \item Ejecutar el proyecto en desarrollo

     \begin{figure}[H]
     \begin{lstlisting}[language=bash]
     npm run dev
     \end{lstlisting}
     \caption{Crear base de datos.}
     \label{fig:create-db}
     \end{figure}

     \item Crear archivo .env el cual nos permetira conectar el front con los demás componentes

     \item Configurar archivo .env con el siguiente texto

     \begin{figure}[H]
     \begin{lstlisting}[language=bash]
 VITE_API_URL=https://fastapi-app-production-d0f4.up.railway.app
 VITE_API_KEY=dev-4b872f28-23f9-4a1e-9a60-1cb7f4090abc
 \end{lstlisting}
 \caption{Configurar .env front}
 \label{fig:config-env-front}
    \end{figure}
\end{enumerate}
