\subsubsection*{A}
\begin{itemize}
    \item \textbf{API (Application Programming Interface):} Conjunto de definiciones y protocolos que permiten la comunicación entre distintas aplicaciones o servicios de software.
    \item \textbf{Administrador del sistema:} Usuario con permisos elevados para gestionar configuraciones, cuentas, dominios y operaciones críticas del sistema.
    \item \textbf{Automatización (OpenAI):} Integración de la API de OpenAI para generar respuestas automáticas a tickets de soporte mediante inteligencia artificial.
\end{itemize}

\subsubsection*{B}
\begin{itemize}
    \item \textbf{Bootstrap:} Framework de diseño front-end que facilita la creación de interfaces web responsivas y consistentes.
    \item \textbf{Bcrypt:} Algoritmo de cifrado utilizado para proteger contraseñas almacenadas de forma segura.
    \item \textbf{Bundler:} Herramienta que agrupa y optimiza archivos JavaScript, CSS y otros recursos para su uso en producción.
\end{itemize}

\subsubsection*{C}
\begin{itemize}
\item \textbf{Carrito de compra:} Funcionalidad que permite a los usuarios seleccionar, almacenar y gestionar productos o servicios antes de completar una compra.



\item \textbf{Cliente (cuenta):} Usuario registrado en el sistema que adquiere y gestiona servicios como dominios y hosting.



\item \textbf{Comisión:} Porcentaje o monto fijo asignado a distribuidores por ventas realizadas dentro del sistema.

\item \textbf{Cuenta (CUENTA):} Registro individual de usuario dentro del sistema, ya sea cliente, distribuidor, empleado o administrador.
\end{itemize}

\subsubsection*{D}
\begin{itemize}
    \item \textbf{DBeaver:} Herramienta de administración de bases de datos que facilita la visualización y edición de datos en entornos SQL.
    \item \textbf{Despliegue:} Proceso de publicación de la aplicación en un entorno accesible
    \item \textbf{Distribuidor:} Usuario con permisos especiales para revender servicios de dominio/hosting, gestionando clientes y comisiones.
    \item \textbf{Dominio (DOMINIO):} Nombre único que identifica un sitio en Internet, adquirido y gestionado por usuarios dentro del sistema.
    \item \textbf{DOMINIOSDisponible:} Tabla o entidad que registra los dominios libres para su compra o asignación.
    \item \textbf{Documentación técnica:} Conjunto de manuales, diagramas y guías destinados a facilitar el desarrollo, uso y mantenimiento del sistema.
\end{itemize}

\subsubsection*{E}
\begin{itemize}
    \item \textbf{Endpoint:} URL específica de una API que permite ejecutar operaciones mediante solicitudes HTTP (GET, POST, etc.).
    \item \textbf{Estado del carrito:} Condición actual del carrito de un usuario (abierto, confirmado, pagado, cancelado, etc.).
    \item \textbf{Estado del ticket:} Seguimiento del progreso de un ticket de soporte (pendiente, en proceso, resuelto, cerrado).
    \item \textbf{Empleado (cuenta):} Usuario con rol operativo dentro del sistema, encargado de soporte o administración técnica.

\end{itemize}

\subsubsection*{F}
\begin{itemize}
    \item {FastAPI:} Framework web para construir APIs en Python, utilizado en el backend por su velocidad y tipado moderno.
    \item {Flask:} Microframework web para Python utilizado para crear la interfaz cliente/administrador y manejar solicitudes HTTP.
    \item {Frontend:} Parte del sistema visible al usuario, desarrollada con tecnologías web como HTML, CSS, JS y React.
\end{itemize}

\subsubsection*{G}
\begin{itemize}
    \item \textbf{Git / GitHub:} Sistema de control de versiones y plataforma colaborativa para alojar el código fuente del sistema.
    \item \textbf{Gestión de tickets:} Funcionalidad que permite registrar, asignar y resolver solicitudes o incidencias de usuarios.
    \item \textbf{Gestión de dominios:} Módulo que permite registrar, transferir o administrar dominios vinculados a los clientes.
    \item \textbf{Gestión de cuentas:} Funcionalidad administrativa para crear, editar, suspender o eliminar cuentas del sistema.
\end{itemize}

\subsubsection*{H}
\begin{itemize}
\item \textbf {Hosting:} Servicio de alojamiento web que permite publicar sitios y aplicaciones en Internet, con diferentes planes disponibles.
\end{itemize}

\subsubsection*{I}
\begin{itemize}
    \item \textbf {Identificación de cuenta:} Proceso que permite determinar el tipo, estado y permisos de una cuenta específica.
    \item \textbf {ItemPaquete:} Unidad o ítem que forma parte de un paquete de hosting, dominio u otro servicio agregado al carrito.
\end{itemize}

\subsubsection*{J}
\begin{itemize}
    \item \textbf {JSX (JavaScript XML):} Sintaxis utilizada en React para escribir componentes con estructura similar a HTML.
\end{itemize}

\subsubsection*{K}

\subsubsection*{L}

\subsubsection*{M}
\begin{itemize}
    \item \textbf {Middleware:} Función o componente intermedio que procesa solicitudes entre el cliente y el backend.
    \item \textbf {Método de pago:} Forma en que un cliente puede abonar por servicios (tarjeta, transferencia, etc.).
    \item \textbf {MySQL:} Sistema de gestión de bases de datos relacional utilizado por la aplicación para almacenar datos estructurados.
\end{itemize}

\subsubsection*{N}
\begin{itemize}
    \item \textbf {NPM (Node Package Manager):} Gestor de paquetes para Node.js, utilizado para instalar dependencias en el frontend.
\end{itemize}

\subsubsection*{O}
\begin{itemize}
    \item \textbf{OpenAI API:} Interfaz de programación utilizada para acceder a modelos de inteligencia artificial generativa de OpenAI.
    \item \textbf{OpenRouter.ai:} Plataforma que permite enrutar solicitudes a múltiples modelos de lenguaje, incluyendo los de OpenAI.
\end{itemize}

\subsubsection*{P}
\begin{itemize}
    \item \textbf{Paquete de hosting:} Conjunto de servicios agrupados bajo un plan que incluye almacenamiento, ancho de banda, correos, etc.
    \item \textbf{Passlib:} Biblioteca de Python utilizada para gestionar cifrado de contraseñas y autenticación segura.
    \item \textbf{Plan (de hosting):} Nivel de servicio contratado, con características técnicas y precio definido.
    \item \textbf{Python:} Lenguaje de programación principal utilizado en el backend del sistema.
\end{itemize}

\subsubsection*{Q}

\subsubsection*{R}
\begin{itemize}
    \item \textbf{React:} Librería de JavaScript para construir interfaces de usuario dinámicas y componentes reutilizables.
    \item \textbf{Registro de cuenta:} Proceso por el cual un nuevo usuario se da de alta en el sistema.
    \item \textbf{ReportLab:} Biblioteca de Python utilizada para generar archivos PDF (por ejemplo, facturas o reportes).
    \item \textbf{Roles de usuario:} Clasificación de cuentas según sus privilegios (cliente, empleado, administrador, distribuidor, etc.).
\end{itemize}

\subsubsection*{S}
\begin{itemize}
    \item \textbf{Segmentación de cuentas:} Clasificación y asignación de cuentas según criterios administrativos o funcionales.
    \item \textbf{Soporte técnico:} Área dedicada a resolver consultas, problemas o solicitudes de los usuarios del sistema.
    \item \textbf{SQLAlchemy:} ORM de Python que facilita la interacción con bases de datos relacionales como MySQL.
    \item \textbf{Sistema de autenticación:} Mecanismo que valida la identidad de usuarios y otorga acceso según sus permisos.
    \item \textbf{Sistema operativo:} Software base que permite la ejecución del sistema, como Windows o Linux.
\end{itemize}

\subsubsection*{T}
\begin{itemize}
\item \textbf{Tarjeta (de pago):} Medio electrónico utilizado por los clientes para abonar productos o servicios.
\item \textbf{Ticket:} Solicitud registrada por un cliente relacionada con soporte técnico, facturación, etc.
\item \textbf{Token:} Cadena generada para identificar sesiones, validar acciones o proteger datos en tránsito.
\item \textbf{Transferencia de dominio:} Proceso mediante el cual un dominio cambia de registrador o de cuenta dentro del sistema.
\item \textbf{Tipo de cuenta:} Clasificación de una cuenta según su función: cliente, empleado, distribuidor, etc.
\item \textbf{Tipo de método de pago:} Categoría asignada al método usado para abonar: tarjeta, efectivo, transferencia.
\end{itemize}

\subsubsection*{U}
\begin{itemize}
\item \textbf{Uvicorn:} Servidor ASGI utilizado para ejecutar aplicaciones web basadas en FastAPI.
\item \textbf{Usuario final:} Persona que utiliza el sistema para contratar servicios y administrar su cuenta.
\end{itemize}

\subsubsection*{V}
\begin{itemize}
\item \textbf{Vite:} Herramienta de desarrollo que acelera la carga y construcción de proyectos frontend modernos.
\item \textbf{Visual Studio Code (VS Code):} Editor de código ampliamente utilizado por desarrolladores para programar y depurar.
\end{itemize}

\subsubsection*{W}
\begin{itemize}
\item \textbf{Web APIs:} Conjunto de servicios accesibles mediante HTTP que permiten la integración entre componentes del sistema.
\item \textbf{Windows 10/11:} Sistemas operativos compatibles con la instalación local y el desarrollo del sistema.
\end{itemize}

\subsubsection*{X}

\subsubsection*{Y}

\subsubsection*{Z}
\begin{itemize}
 \item \textbf{Zed:} Editor de código de Linux utilizado por desarrolladores para programar y depurar.
\end{itemize}
