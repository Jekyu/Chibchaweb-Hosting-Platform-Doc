Para el desarrollo del sistema de información de ChibchaWeb, se ha seleccionado la metodología ágil Scrum. Esta elección se basa en las características del proyecto y las necesidades del cliente, que requieren flexibilidad, entregas parciales funcionales y un involucramiento constante de los interesados en el proceso.

\subsection{Scrum}
Scrum es un marco ágil de desarrollo iterativo e incremental que permite entregar valor de manera temprana y continua a través de sprints (iteraciones de tiempo fijo, típicamente de 1 a 4 semanas).\\
En Scrum, el trabajo se organiza en:

\begin{itemize}
    \item \textbf{Product Backlog:} Lista priorizada de funcionalidades y requerimientos del sistema.
    \item \textbf{Sprint Backlog:} Conjunto de tareas seleccionadas para el Sprint en curso.
    \item \textbf{Sprint Review y Sprint Retrospective:} Evaluaciones al finalizar cada sprint, enfocadas en la mejora continua.
\end{itemize}

\subsubsection{Roles en Scrum aplicados al proyecto}

\begin{itemize}
    \item \textbf{Product Owner:} Representa los intereses de ChibchaWeb. Define y prioriza el Product Backlog.
    \item \textbf{Scrum Master:} Facilita el marco de trabajo Scrum, elimina impedimentos y promueve buenas prácticas.
    \item \textbf{Development Team:} Grupo autoorganizado de desarrolladores encargados de diseñar, codificar, probar e implementar el sistema.
\end{itemize}

\subsection{Justificación de su uso en este proyecto}

\begin{itemize}
    \item \textbf{Adaptabilidad a cambios de requisitos:} Dado que el negocio de hospedaje web evoluciona rápidamente (nuevas formas de pago, expansión internacional), Scrum permite adaptarse con agilidad.
    \item \textbf{Entrega continua de valor:} Con cada sprint, se entregan versiones funcionales del sistema que pueden ser probadas por el cliente.
    \item \textbf{Comunicación constante con el cliente:} La retroalimentación frecuente mejora la alineación entre lo que se construye y lo que el cliente realmente necesita.
    \item \textbf{Reducción del riesgo:} La entrega incremental permite detectar errores o desviaciones de manera temprana.
    \item \textbf{Fomento de la colaboración y mejora continua:} Las reuniones retrospectivas y la autoorganización del equipo fortalecen la eficiencia del proceso.
\end{itemize}
