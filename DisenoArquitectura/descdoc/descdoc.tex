\subsection{Propósito y Audiencia}
Este documento tiene como propósito describir la arquitectura de software propuesta para el desarrollo de una solución integral que apoye los procesos operativos de ChibchaWeb, una compañía dedicada al hospedaje de páginas web con sede en Sugamuxi y proyección de expansión internacional.\\
El documento presenta los componentes clave del sistema, los requerimientos funcionales y no funcionales, los patrones de diseño seleccionados, y los modelos arquitectónicos que sustentan la solución. También detalla la metodología de desarrollo adoptada, las decisiones técnicas tomadas y su justificación.\\
Este documento está dirigido a los siguientes perfiles:

\begin{itemize}
    \item{Equipo de Desarrollo: Para comprender la arquitectura, los patrones y las tecnologías definidas que guiarán la implementación del sistema.}
    \item{Clientes y Stakeholders de ChibchaWeb: Para validar que los requerimientos funcionales y no funcionales del negocio han sido entendidos y serán satisfechos por el sistema.}
    \item{Equipo de QA y Soporte Técnico: Para conocer los aspectos técnicos relevantes de la solución y planificar las estrategias de prueba y soporte.}
    \item{Líderes de Proyecto y Gerencia: Para tomar decisiones informadas sobre tiempos, recursos y alineación con objetivos estratégicos.}
\end{itemize}

\subsection{Organización del Documento}
Este documento empieza con un panorama general del proyecto consistente en objetivos generales y específicos, seguidos de una contextualización del entorno y conceptos importante del proyecto, esto con el fin de generar un marco teórico que aclare y mejore la organización empresarial.\\
Luego de esto, se define el modelo de desarrollo y por lo tanto la organización del equipo de desarrollo. Después se tiene en cuenta las motivaciones y los riesgos relacionados con el desarrollo del proyecto, con el fin de estimar la utilidad. Además de las restricciones (tecnológico, negocio) y la gestión de calidad (con parámetros medibles para cumplir objetivos).\\
Por último, se definen los diagramas principales para empezar el proceso del producto, describiendo los diferentes modelos y entidades del desarrollo.
Para terminar, se hace una pequeña retroalimentación con el fin de reconocer errores dentro del desarrollo del proyecto y corregirlos en próximos proyectos.
\subsection{Convenciones}

\subsubsection*{Identificadores de requerimientos:}
\begin{itemize}
	\item Requerimientos funcionales: RF-XX (ej. RF-01).
\end{itemize}


\subsection{Terminología y Definiciones}

\begin{itemize}
\item Cliente: Usuario registrado que adquiere servicios de hosting o dominios.
\item Distribuidor: Usuario que intermedia la venta de dominios y recibe comisiones.
\item Empleado: Usuario interno con funciones operativas dentro de ChibchaWeb.
\item Administrador: Usuario con permisos totales de gestión.
\item Ticket: Registro de solicitud de soporte técnico.
\item Dominio: Nombre único utilizado para identificar un sitio web en Internet.
\item Carrito: Módulo de compra donde se almacenan temporalmente productos antes del pago.
\item Scrum: Marco de trabajo ágil usado para el desarrollo del proyecto.
\item Frontend: Parte de la aplicación con la que interactúa el usuario final.
\item Backend: Parte de la aplicación que maneja la lógica de negocio y la comunicación con la base de datos.
\item PSE: Plataforma de pagos electrónicos utilizada en Colombia.
\item XML: Lenguaje de marcado utilizado para intercambio de información estructurada.
\item Railway: Plataforma de despliegue utilizada para alojar el sistema.
\item Scraping: Técnica para extraer información de páginas web de terceros.
\item Método de pago: Forma registrada por el usuario para realizar transacciones (tarjeta de crédito, PSE, etc.).
\end{itemize}
