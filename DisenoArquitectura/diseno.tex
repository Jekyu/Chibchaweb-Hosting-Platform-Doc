\documentclass[a4paper, 12pt]{article}
\usepackage[utf8]{inputenc}
\usepackage[spanish, provide=*]{babel}
\usepackage{amsmath, amssymb}
\usepackage{graphicx}
\usepackage{hyperref}
\usepackage{listings}
\usepackage{tcolorbox}
\usepackage{xcolor}
\usepackage{fancyhdr}
\usepackage{geometry}

\geometry{left=2.5cm, right=2.5cm, top=3cm, bottom=4.5cm}

% Para las tablas
\usepackage{float}  %Para que no se cambien de lugar
\usepackage{pgfplotstable}
\usepackage{array}
\usepackage{colortbl}
\usepackage{url} % para que el campo note muestre URLs bien

% Estilo
\usepackage{lastpage}

% Definición de colores
\definecolor{cafetitulos}{RGB}{150,75,0}
\definecolor{redv}{RGB}{165, 43, 42}
\definecolor{cafe}{RGB}{130,75,0}
\definecolor{cafet}{RGB}{204, 121, 5}

% Configuración de encabezados y pies de página
\pagestyle{fancy}
\renewcommand{\headrulewidth}{0pt} % Elimina la línea por defecto del encabezado

% Encabezado
\fancyhead[L]{\includegraphics[height=1.5cm]{logomanual.png}}
\fancyhead[R]{\nouppercase{\leftmark}} % Nombre de la sección

% Pie de página
\renewcommand{\footrulewidth}{0.5pt} % Línea separadora
\renewcommand{\footrule}{\hbox to\headwidth{\color{cafe}\leaders\hrule height \footrulewidth\hfill}}
\fancyfoot[L]{Documento de Diseño y Arquitectura - 2025}
\fancyfoot[C]{}
\fancyfoot[R]{Página \thepage} % Número de página a la derecha

% Ajustar altura del encabezado y espacio para el pie
\setlength{\headheight}{52pt} % Altura del encabezado (ajusta según el logo)
\setlength{\footskip}{40pt} % Espacio entre el texto y el pie (evita solapamiento)

% Titulos
\usepackage{titlesec} % Para personalizar section/subsection
\titleformat{\section}
  {\normalfont\Large\bfseries}{\thesection}{1em}{\color{cafetitulos}} % Section en café
\titleformat{\subsection}
  {\normalfont\large\bfseries}{\thesubsection}{1em}{\color{cafe}}
\titleformat{\subsubsection}{\normalfont\bfseries}{\thesubsubsection}{1em}{\color{cafetitulos}}

% Configuración para código fuente
\lstset{
    language=Java, % puedes cambiar a Python, SQL, etc.
    basicstyle=\ttfamily\small,
    backgroundcolor=\color{gray!10},
    frame=single,
    breaklines=true,
    keywordstyle=\color{cafetitulos},
    commentstyle=\color{gray},
    stringstyle=\color{redv}
}


\title{
\includegraphics[width=5cm]{logo.png}\\

{\Large CHIBCHAWEB HOSTING PLATFORM}\\
{\Large APLICACIÓN PARA UNA COMPAÑÍA DE HOSPEDAJE DE PÁGINAS WEB}\\[0.5em]

\textbf{DOCUMENTO DE DISEÑO Y ARQUITECTURA}\\[0.5em]
}
\author{
\textbf{Grupo A}\\[0.5em]
Ingenieros:\\
Andrés José Acevedo Ardila\\
Brayan Steven Sánchez Casilima\\
Christian Camilo Lancheros Sánchez\\
Juan Esteban Ordoñez Sandoval\\
Karen Tatiana Bravo Rodríguez\\
Luis Felipe Mayorga Tibaquicha\\\\\\
Universidad Distrital Francisco José de Caldas\\
Facultad de Ingeniería\\
Ingeniería de sistemas\\
}
\date{Bogotá D.C., Agosto 2025}


% Inicio Documento
\begin{document}
\maketitle

\newpage
\begin{titlepage}
\thispagestyle{empty}

\begin{tikzpicture}[remember picture, overlay]
    \node[anchor=north east, inner sep=0pt, xshift=13cm] at (current page.north east) {
        \includegraphics[height=\paperheight]{registro-imagen.png}
    };
\end{tikzpicture}

\vfill
{\includegraphics[width=0.8\textwidth]{logomanual.png}\par}
\vfill
{\Huge Grupo A\par}
{\Huge Documento de Diseño
y Arquitectura\par}
{\Large 2025\par}
\vfill
\restoregeometry

\end{titlepage}


\newpage
\phantomsection
\tableofcontents

\newpage
\section{Listado de Figuras}
\listoffigures

\newpage
\section{Listado de Tablas}
\listoftables

\newpage
\section{Descripción del Documento}
\subsection{Propósito y Audiencia}
Este documento tiene como propósito describir la arquitectura de software propuesta para el desarrollo de una solución integral que apoye los procesos operativos de ChibchaWeb, una compañía dedicada al hospedaje de páginas web con sede en Sugamuxi y proyección de expansión internacional.\\
El documento presenta los componentes clave del sistema, los requerimientos funcionales y no funcionales, los patrones de diseño seleccionados, y los modelos arquitectónicos que sustentan la solución. También detalla la metodología de desarrollo adoptada, las decisiones técnicas tomadas y su justificación.\\
Este documento está dirigido a los siguientes perfiles:

\begin{itemize}
    \item{Equipo de Desarrollo: Para comprender la arquitectura, los patrones y las tecnologías definidas que guiarán la implementación del sistema.}
    \item{Clientes y Stakeholders de ChibchaWeb: Para validar que los requerimientos funcionales y no funcionales del negocio han sido entendidos y serán satisfechos por el sistema.}
    \item{Equipo de QA y Soporte Técnico: Para conocer los aspectos técnicos relevantes de la solución y planificar las estrategias de prueba y soporte.}
    \item{Líderes de Proyecto y Gerencia: Para tomar decisiones informadas sobre tiempos, recursos y alineación con objetivos estratégicos.}
\end{itemize}

\subsection{Organización del Documento}
Este documento empieza con un panorama general del proyecto consistente en objetivos generales y específicos, seguidos de una contextualización del entorno y conceptos importante del proyecto, esto con el fin de generar un marco teórico que aclare y mejore la organización empresarial.\\
Luego de esto, se define el modelo de desarrollo y por lo tanto la organización del equipo de desarrollo. Después se tiene en cuenta las motivaciones y los riesgos relacionados con el desarrollo del proyecto, con el fin de estimar la utilidad. Además de las restricciones (tecnológico, negocio) y la gestión de calidad (con parámetros medibles para cumplir objetivos).\\
Por último, se definen los diagramas principales para empezar el proceso del producto, describiendo los diferentes modelos y entidades del desarrollo.
Para terminar, se hace una pequeña retroalimentación con el fin de reconocer errores dentro del desarrollo del proyecto y corregirlos en próximos proyectos.
\subsection{Convenciones}
\subsection{Terminología y Definiciones}


\newpage
\section{Generalidades del proyecto}
\subsection{Problema por resolver}
ChibchaWeb, una empresa de hospedaje de sitios web ubicada en Sugamuxi, ha experimentado un crecimiento constante en su base de clientes, principalmente en Colombia y países cercanos, y se proyecta una expansión hacia otros continentes como África. Actualmente administra miles de sitios web bajo distintos paquetes de hosting (Chibcha-Platino, Chibcha-Plata y Chibcha-Oro) sobre plataformas Windows y Unix. Los clientes pueden escoger planes de pago flexibles (mensual, trimestral, semestral o anual), y aunque el método principal de pago es con tarjeta de crédito, se planea habilitar próximamente otros medios como PSE.

El modelo de negocio incluye además una red de distribuidores categorizados como BÁSICOS o PREMIUM, con diferentes esquemas de comisiones. ChibchaWeb también intermedia en el registro y transferencia de dominios a través de terceros registradores, ofreciendo este servicio adicional a sus clientes. A medida que el volumen de operaciones crece y los procesos se diversifican, la compañía enfrenta varios desafíos operativos y administrativos.

Actualmente, ChibchaWeb carece de una solución unificada que automatice y centralice sus procesos comerciales, operativos y de atención al cliente. Esto genera ineficiencias en la gestión de perfiles, pagos, comisiones, atención técnica y comunicación con registradores externos. La falta de un sistema integral limita la capacidad de escalar operaciones, introducir nuevos métodos de pago, y garantizar trazabilidad y control sobre los procesos internos.

\subsection{Descripción general del sistema a desarrollar}
El sistema a desarrollar para ChibchaWeb es una plataforma integral de gestión de servicios de hospedaje web, diseñada para automatizar, centralizar y optimizar los procesos operativos clave de la compañía. Esta solución tecnológica permitirá gestionar clientes, empleados, distribuidores, dominios, pagos y soporte técnico de manera eficiente, segura y escalable.

El sistema contará con módulos para la creación y administración de perfiles de clientes y empleados, y almacenará la información en una base de datos relacional optimizada incluyendo la validación de información personal y financiera garantizando la integridad evitando ciclos en esta; buscar y adquirir dominios mediante un carrito; adquirir paquetes de hosting.\\

El sistema contará con roles diferenciados por tipo de cuenta; búsquedas y consultas detalladas de todos los actores del sistema, así como herramientas para la generación de archivos de transferencia electrónica y seguimiento de actividades internas del proyecto; gestión de solicitudes de dominios con generación de archivos XML para su envío a registradores externos; Cálculo automático de comisiones a distribuidores según su categoría (BÁSICO o PREMIUM); y un módulo de soporte técnico basado en tickets que garantiza el seguimiento y resolución estructurada de problemas reportados.\\

La solución incluirá una interfaz web en React, backend en FastAPI y base de datos en MySQL. Se aplicarán buenas prácticas de diseño para mantener una l ógica de negocio clara y una estructura coherente en el modelo de datos.

\subsection{Objetivos de la solución}

\subsubsection{Objetivo general}
\begin{itemize}
	\item Desarrollar un sistema web que permita registrar, gestionar y monitorear cuentas de usuarios, adquisición de dominios y facturación de manera eficiente y centralizada.

\end{itemize}

\subsubsection{Objetivos específicos}
\begin{itemize}
\item Diseñar una base de datos relacional normalizada y sin ciclos innecesarios para soportar la lógica del sistema.
\item Desarrollar una interfaz web funcional y responsiva utilizando React y Bootstrap para facilitar la interacción del usuario con el sistema.
\item Construir un backend en FastAPI que gestione la lógica del negocio, la autenticación de usuarios y la comunicación con la base de datos.
\end{itemize}

\subsection{Stakeholders}
%configutación tablas
\pgfplotstableset{
    % Estilo general
    every head row/.style={
        before row=\hline\rowcolor{cafet}\bfseries,
        after row=\hline,
    },
    every odd row/.style={before row=\rowcolor{gray!10}},
    columns/Stakeholder/.style={string type, column type=p{4cm}},
    columns/Descripción/.style={string type,column type=p{12cm},},
    col sep=semicolon,
}

\begin{center}
    \begin{table}[H]
    \pgfplotstabletypeset[
    col sep=semicolon,
    header=true,
    ]{generalidades/stakeholders.dat}
    \caption{Stakeholders}
    \end{table}
\end{center}


\newpage
\section{Descripción y justificación de la metodología}
Para el desarrollo del sistema de información de ChibchaWeb, se ha seleccionado la metodología ágil Scrum. Esta elección se basa en las características del proyecto y las necesidades del cliente, que requieren flexibilidad, entregas parciales funcionales y un involucramiento constante de los interesados en el proceso.

\subsection{Scrum}
Scrum es un marco ágil de desarrollo iterativo e incremental que permite entregar valor de manera temprana y continua a través de sprints (iteraciones de tiempo fijo, típicamente de 1 a 4 semanas).\\
En Scrum, el trabajo se organiza en:

\begin{itemize}
    \item \textbf{Product Backlog:} Lista priorizada de funcionalidades y requerimientos del sistema.
    \item \textbf{Sprint Backlog:} Conjunto de tareas seleccionadas para el Sprint en curso.
    \item \textbf{Sprint Review y Sprint Retrospective:} Evaluaciones al finalizar cada sprint, enfocadas en la mejora continua.
\end{itemize}

\subsubsection{Roles en Scrum aplicados al proyecto}

\begin{itemize}
    \item \textbf{Product Owner:} Representa los intereses de ChibchaWeb. Define y prioriza el Product Backlog.
    \item \textbf{Scrum Master:} Facilita el marco de trabajo Scrum, elimina impedimentos y promueve buenas prácticas.
    \item \textbf{Development Team:} Grupo autoorganizado de desarrolladores encargados de diseñar, codificar, probar e implementar el sistema.
\end{itemize}

\subsection{Justificación de su uso en este proyecto}

\begin{itemize}
    \item \textbf{Adaptabilidad a cambios de requisitos:} Dado que el negocio de hospedaje web evoluciona rápidamente (nuevas formas de pago, expansión internacional), Scrum permite adaptarse con agilidad.
    \item \textbf{Entrega continua de valor:} Con cada sprint, se entregan versiones funcionales del sistema que pueden ser probadas por el cliente.
    \item \textbf{Comunicación constante con el cliente:} La retroalimentación frecuente mejora la alineación entre lo que se construye y lo que el cliente realmente necesita.
    \item \textbf{Reducción del riesgo:} La entrega incremental permite detectar errores o desviaciones de manera temprana.
    \item \textbf{Fomento de la colaboración y mejora continua:} Las reuniones retrospectivas y la autoorganización del equipo fortalecen la eficiencia del proceso.
\end{itemize}


\newpage
\section{Especificación y Recabación de Requerimientos Funcionales}
La presente sección describe los requerimientos funcionales necesarios para el desarrollo de la aplicación de software para ChibchaWeb, con el fin de automatizar y optimizar los procesos de gestión de clientes, pagos, dominios, distribuidores y soporte técnico. Los requerimientos han sido identificados a partir del análisis de las operaciones actuales de la empresa y sus proyecciones.


\section{Motivadores Arquitecturales}
\subsection{Motivadores de Negocio}
Los motivadores de negocio que se mostrarán a continuación definen las razones por las cuales ChibchaWeb requiere una nueva solución de software. Entre las cuales se encuentran:

\subsubsection{Expansión internacional del negocio}
ChibchaWeb, actualmente con clientes en Colombia y países vecinos, proyecta una expansión hacia mercados internacionales como África. Esto requiere una plataforma escalable y flexible que soporte la gestión de clientes, dominios y facturación en diferentes regiones.

\subsubsection{Automatización de procesos clave}
El sistema busca automatizar la creación, edición y eliminación de perfiles de clientes, empleados y distribuidores, reduciendo la carga operativa y los errores humanos, y facilitando la trazabilidad de operaciones críticas.

\subsubsection{Mejora en la atención al cliente y soporte técnic}

Al permitir el registro y seguimiento de tickets, el sistema mejorará el proceso de resolución de problemas, permitiendo al equipo de soporte priorizar incidencias y mejorar los tiempos de respuesta, alineándose con los niveles de servicio ofrecidos.

\subsubsection{Gestión y segmentación de distribuidores}
Con la existencia de distribuidores BÁSICOS y PREMIUM, el sistema debe permitir su categorización, cálculo de comisiones correspondientes (10% o 15%), y la generación de reportes financieros para pagos electrónicos, fomentando relaciones sólidas con aliados estratégicos.

\subsubsection{Flexibilidad en métodos de pago}
Actualmente se aceptan tarjetas de crédito, pero se planea incluir otros métodos como PSE. Esto requiere una arquitectura extensible que permita integrar fácilmente nuevos métodos de pago en el futuro sin afectar la operatividad actual.

\subsection{Restricciones de Tecnología}
El desarrollo de la plataforma para ChibchaWeb estará sujeto a las siguientes restricciones tecnológicas, derivadas de decisiones de arquitectura, compatibilidad, mantenibilidad y recursos disponibles:

Tecnologías definidas por el equipo de desarrollo

 \subsubsection{Frontend}

Se utilizará React.js como framework principal para la construcción de la interfaz de usuario, debido a su eficiencia, modularidad y gran comunidad de soporte.

\subsubsection{Backend}

Se empleará FastAPI con Python, por su rendimiento, facilidad de definición de servicios RESTful, y compatibilidad con documentación automática.

\subsubsection{Base de datos}

Se usará MySQL, una base de datos relacional robusta y ampliamente soportada, ideal para mantener la integridad referencial y relaciones entre entidades complejas (clientes, empleados, dominios, etc.).

\subsubsection{Control de versiones y colaboración}

El proyecto estará versionado mediante GitHub, lo cual permitirá control de cambios, trabajo colaborativo y despliegue continuo si se requiere.

\subsection{Restricciones de Negocio}
El desarrollo y funcionamiento del sistema estarán sujetos a diversas restricciones impuestas por las necesidades, normativas y condiciones del entorno comercial del proyecto. Estas restricciones afectan la manera en que se diseña y entrega la solución, e incluyen:

\subsubsection{Políticas de manejo de cuentas }

Solo se permitirán 4 tipos de cuentas: cliente, administrador, empleado y distribuidor, cada una con permisos y accesos claramente definidos. No se admitirán roles personalizados fuera de estas categorías durante la fase inicial del sistema.

\subsubsection{Manejo de dominios basado en disponibilidad externa }

La compra de dominios dependerá de su disponibilidad en registros externos. Si el dominio ya está tomado, no podrá ser adquirido ni facturado, lo cual impone una dependencia con proveedores externos y puede requerir lógica de validación en tiempo real.

\subsection{Atributos de Calidad}

Durante el desarrollo del sistema ChibchaWeb, se identificaron ciertos atributos de calidad que orientaron decisiones técnicas y de arquitectura. A continuación, se describen los principales, en función de la experiencia práctica con el sistema ya desplegado y en funcionamiento.

\subsubsection{Disponibilidad}

El sistema actualmente está desplegado mediante Railway, por lo tanto, su disponibilidad depende del servicio que ofrece esta plataforma. No se cuenta con una infraestructura propia, pero la integración entre base de datos, backend y frontend se mantiene activa y estable mientras Railway esté operativo. Esta disponibilidad es suficiente para los fines actuales, aunque si en un futuro fuera necesario se podría considerar una solución con mayor control sobre el entorno de producción.

\subsubsection{Seguridad}

Se implementó un sistema de autenticación por correo electrónico, en el cual los usuarios reciben un código de verificación. Solo quienes completen este proceso pueden acceder a las funciones reservadas para clientes. Esta medida ha resultado efectiva para restringir accesos y proteger la información del usuario, y se aplican validaciones en los formularios.


\subsubsection{Rendimiento}

La aplicación responde de forma fluida gracias a la conexión directa con las APIs del backend. No se ven demoras significativas al realizar operaciones como búsquedas, registros, etc.


\subsubsection{Escalabilidad}

Actualmente el sistema permite operaciones como búsqueda de dominios y la “compra” como tal, lo cual es una base para un entorno escalable. Aunque todavía no se ofrecen servicios de hosting reales desde la plataforma, el sistema está preparado para incluir esa funcionalidad más adelante, como tal el nucleo del sistema nos permite esto


\subsubsection{Mantenibilidad}

El código está organizado de forma clara, permitiendo modificar o añadir funcionalidades sin afectar lo ya implementado. Se han realizado ajustes menores y correcciones sin mayores inconvenientes, lo que demuestra que el sistema es mantenible. Además, se pueden actualizar componentes del frontend o backend de manera independiente.


\subsubsection{Usabilidad}

Desde la perspectiva del cliente, la interfaz es bastante intuitiva. Las funciones están bien distribuidas y no es necesario tener conocimientos técnicos para navegar o realizar operaciones básicas. En el caso del personal de soporte o empleados, hay más funciones que requieren cierta formación, como el manejo de tickets, pero están diseñadas de forma clara.


\subsubsection{Interoperabilidad}

El sistema se comunica con servicios externos mediante peticiones web, especialmente para consultar la disponibilidad de dominios. Para ello, se implementó un proceso de scraping que permite validar si un dominio está disponible o no.


\subsubsection{Recabación y justificació de Requerimientos No Funcionales}
Durante la planificación y construcción del sistema ChibchaWeb, se identificaron diversos requerimientos no funcionales a partir de las necesidades específicas del proyecto, la experiencia con despliegues reales, y las condiciones técnicas impuestas por el entorno en el que opera la aplicación. Estos requerimientos no son visibles directamente para el usuario final, pero son determinantes para que el sistema sea confiable, seguro y sostenible en el tiempo.

\begin{itemize}
    \item {Las operaciones realizadas por el usuario (como la búsqueda de dominios, la carga de formularios o la visualización de datos) no es de un tiempo significativo o excesivo, esto mejora la experiencia del usuario, esto gracias al uso de las Apis }
    \item {Aunque actualmente se encuentra todo en el servicio de railway para el despligue, fuera de eso el sistema es funcional y estable. }
    \item {Solo los usuarios verificados deben tener acceso a funcionalidades sensibles del sistema. Además, los datos sensibles  como contraseña están protegidos  }
    \item {El sistema esta preparado para soportar un aumento progresivo de usuarios y funcionalidades sin comprometer el rendimiento, aunque recalcar que al menos en esta fase la disponibilidad no depende tanto de nosotros, sino del servicio de railway }
    \item {En diferentes momentos del proyecto se han realizado cambios o mejoras con relativa facilidad, lo que indica que se ha logrado un diseño mantenible. }
    \item {Ya se implementó una primera forma de integración con servicios de disponibilidad de dominios usando scraping. Esta capacidad de interactuar con sistemas externos es la clave para estas funcionalidades. }
\end{itemize}

\subsubsection{Escenarios de Calidad}
Los escenarios de calidad permiten evaluar cómo se comporta el sistema ante situaciones específicas que afectan atributos no funcionales como el rendimiento, la seguridad o la disponibilidad. A continuación se presentan algunos escenarios relevantes que reflejan condiciones reales o esperadas para la plataforma ChibchaWeb:

\subsubsection*{Consulta masiva de dominios disponibles}

\begin{itemize}
\item \textbf{Contexto:} Un distribuidor accede a la plataforma y realiza varias búsquedas consecutivas para verificar la disponibilidad de dominios para sus clientes.

\item \textbf{Estimulación:} Se ejecutan 10 a 15 búsquedas en un corto periodo de tiempo.

\item \textbf{Respuesta esperada:} El sistema podra responder a estas consultas en un tiempo no tan significativo, sin errores ni bloqueos.
\end{itemize}

\subsubsection*{Intento de acceso sin verificación de correo}

\begin{itemize}
\item \textbf{Contexto:} Un usuario se registra pero no verifica su correo electrónico.

\item \textbf{Estimulación:} Intenta acceder a las funcionalidades de cliente como compra de dominio o visualización del perfil.

\item \textbf{Respuesta esperada:} El sistema bloquea el acceso y solicita la verificación de correo antes de continuar.
\end{itemize}

 \subsubsection*{Actualización del backend sin detener el sistema completo}

\begin{itemize}
\item \textbf{Contexto:} Se implementa una mejora en una API del backend.

\item \textbf{Estimulación:} Se despliega el nuevo backend mientras la base de datos y el frontend siguen activos.

\item \textbf{Respuesta esperada:} El sistema sigue funcionando normalmente o se interrumpe solo por un corto tiempo. No se generan inconsistencias.
\end{itemize}

\subsubsection*{Interacción de un cliente no técnico}

\begin{itemize}
\item \textbf{Contexto:} Un cliente nuevo accede al sistema para registrar su primer dominio.

\item \textbf{Estimulación:} Usa la interfaz para buscar un dominio, registrar sus datos y simular un pago.

\item \textbf{Respuesta esperada:} El cliente puede completar la operación sin necesidad de asistencia. Los pasos son claros y comprensibles.
\end{itemize}

\subsubsection*{Caída temporal del servicio Railway}

\begin{itemize}
\item \textbf{Contexto:} Railway presenta una interrupción temporal.

\item \textbf{Estimulación:} Usuarios intentan ingresar al sistema durante el tiempo en que el servicio está inactivo.

\item \textbf{Respuesta esperada:} El sistema no responde (fuera de servicio), pero se recupera automáticamente cuando Railway vuelve a estar disponible, sin pérdida de información.
\end{itemize}


\newpage
\section{Contexto}
\subsection{Escenarios operacionales}
Estos Escenarios operacionales representan las interacciones comunes o típicas que pasan entre los diferentes actores de la plataforma de ChibchaWeb, Cada escenario describe como un usuario o actor logra el objetivo deseado utilizando el sistema en un contexto especifico

\subsubsection{Escenario 1: Registro y verificación de cuenta}

\begin{itemize}
\item \textbf{Actor:} Usuario
\item \textbf{Descripción:} Un usuario se registra en la plataforma ingresando sus datos personales. Luego, valida su cuenta a través de un token enviado por correo electrónico.
\item \textbf{Resultado esperado:} El usuario puede iniciar sesión y acceder a las funciones de su rol.
\end{itemize}

\subsubsection{Escenario 2: Compra de un paquete de hosting}

\begin{itemize}
\item \textbf{Actor:} Cliente
\item \textbf{Descripción:} El cliente agrega un método de pago, selecciona un paquete, realiza el pago y se activa el servicio.
\item \textbf{Resultado esperado:} El sistema registra la compra.
\end{itemize}

\subsubsection{Escenario 3: Registro de solicitud de dominio}

\begin{itemize}
\item \textbf{Actor:} Usuario (Cliente o Distribuidor)
\item \textbf{Descripción:} El usuario busca un dominio, valida su disponibilidad y envía una solicitud.
\item \textbf{Resultado esperado:} La solicitud es procesada y almacenada para generación de archivos XML.
\end{itemize}

\subsubsection{Escenario 4: Atención de ticket de soporte}

\begin{itemize}
\item \textbf{Actor:} Usuario
\item \textbf{Descripción:} El usuario crea un ticket describiendo su problema. Soporte lo atiende, lo puede escalar y responder. El administrador puede asignarlo.
\item \textbf{Resultado esperado:} se le da solución al ticket lo más pronto posible
\end{itemize}



\subsubsection{Escenario 5: Gestión administrativa de usuarios}

\begin{itemize}
\item \textbf{Actor:} Administrador
\item \textbf{Descripción:} Desde el panel de administración, el administrador puede gestionar precios, paquetes y consultar
\item \textbf{Resultado esperado:} Los cambios se reflejan en la base de datos y actualizan el aplicativo en tiempo real
\end{itemize}



\subsubsection{Escenario 6: Cálculo y pago de comisiones}

\begin{itemize}
\item \textbf{Actor:} Administrador
\item \textbf{Descripción:} El sistema calcula las comisiones para los distribuidores según su categoría básico o premium (recordar que esto es un descuento)
\item \textbf{Resultado esperado:} El administrador tiene listos los valores y la documentación de soporte.
\end{itemize}

\subsubsection{Escenario 7: Consulta de perfil personal}

\begin{itemize}
\item \textbf{Actor:} Usuario
\item \textbf{Descripción:} El usuario inicia sesión, accede a su perfil y visualiza información relevante como nombre, correo, y acceso a sus métodos de pago o registrar uno nuevo
\item \textbf{Resultado esperado:} El usuario tiene acceso a un panel completo sobre sus datos de forma clara
\end{itemize}

\subsubsection{Escenario 8: Transferencia de dominio vía correo}

\begin{itemize}
\item \textbf{Actor:} Usuario
\item \textbf{Descripción:} Un usuario puede transferir un dominio en su propiedad al correo de otro usuario.
\item \textbf{Resultado esperado:} La transferencia queda registrada y se notifica a los usuarios involucrados
\end{itemize}

\subsection{Casos de Uso: Descripción y Modelo}

Los casos de uso del sistema ChibchaWeb describen las funcionalidades clave que pueden ser ejecutadas por los distintos actores, permitiendo alcanzar objetivos específicos mediante la interacción con la aplicativo. Este modelo facilita la comprensión de los límites del sistema, las responsabilidades de cada actor y las relaciones entre funcionalidades.\\

Actores Del Sistema:
\begin{itemize}
    \item \textbf{Usuario:} Actor general del que heredan todos los demás.
    \item \textbf{Cliente:} Usuario registrado que puede adquirir servicios y registrar dominios.
    \item \textbf{Distribuidor:} Usuario que gestiona múltiples dominios y cobra a sus propios clientes.
    \item \textbf{Empleado:} Usuario interno con funcione operativas.
    \item \textbf{Soporte:} Subtipo de empleado, encargado de atender tickets.
    \item \textbf{Administrador:} Usuario con permisos totales para gestionar perfiles, dominios y comisiones
\end{itemize}

\subsubsection{Resumen De Casos De Uso}

\pgfplotstableset{
    % Estilo general
    every head row/.style={
        before row=\hline\rowcolor{cafet}\bfseries,
        after row=\hline,
    },
    every odd row/.style={before row=\rowcolor{gray!10}},
    columns/Categoría/.style={string type, column type=p{4cm}},
    columns/Caso de Uso/.style={string type, column type=p{8cm}},
    columns/Actor(es)/.style={string type,column type=p{3cm},},
    col sep=semicolon,
}

\begin{center}
    \begin{table}[H]
    \pgfplotstabletypeset[
    col sep=semicolon,
    header=true,
    ]{contexto/rescas.dat}
    \caption{Resumen De Casos De Uso}
    \end{table}
\end{center}


\newpage
\section{Descripción y justificación de patrones de diseño adoptados y arquitectura de software definida.}
En el desarrollo de la plataforma ChibchaWeb se implementaron diversos patrones de diseño que permitieron mantener la claridad en la arquitectura, separación de responsabilidades y facilidad de mantenimiento del código.

\subsubsection{Patrón DTO (Data Transfer Object)}
Utilizado para la validación y transporte de datos entre el frontend y el backend. Se implementó mediante Pydantic en FastAPI, garantizando que la información recibida cumpla con el formato esperado antes de ser procesada, reduciendo así errores y mejorando la seguridad de entrada de datos.

\subsubsection{Patrón Active Record}
Aplicado en los modelos de datos mediante SQLAlchemy, permitiendo que cada entidad del sistema (Cliente, Dominio, Paquete, etc.) contenga tanto su representación como las operaciones CRUD asociadas, simplificando el acceso y manipulación de la base de datos.

\subsubsection{Patrón Controller}
Implementado a través de los routers de FastAPI, donde cada módulo del sistema cuenta con controladores que gestionan las solicitudes HTTP, orquestan la lógica de negocio y retornan las respuestas correspondientes.

\subsubsection{Encapsulación de lógica en servicios}
Funcionalidades específicas, como el generador de respuestas mediante inteligencia artificial, se desarrollaron como servicios independientes. En este componente se emplearon los patrones Strategy (para permitir distintas formas de generar respuestas según el contexto) y Adapter (para integrar proveedores externos de IA con la lógica interna del sistema).

\subsubsection{Patrón Template Method}
Los flujos de facturación y la generación de solicitudes XML para registradores externos se estructuraron siguiendo este patrón, definiendo la secuencia general de pasos y permitiendo que subclases concreten detalles específicos, lo que facilita cambios futuros sin modificar la estructura base del proceso.

En conjunto, la aplicación de estos patrones ha permitido desarrollar un sistema modular, escalable y adaptable a los cambios, manteniendo buenas prácticas de ingeniería de software y asegurando la coherencia entre los distintos componentes.


\newpage
\section{Puntos de Vista y Modelos Arquitecturales}
\subsection{Punto de Vista: Descripción del problema.}
Para describir el problema a solucionar al desarrollar la plataforma propuesta, se presentan a continuación dos diagramas para describir el punto de vista de contexto de la solución:

\subsubsection{Punto de Vista: Contexto de la solución}
Donde se aprecia el sistema a desarrollar (ChibchaWeb Hosting Platfom) y las entradas y salidas que posee el sistema:

\begin{figure}[H]
    \centering
	\includegraphics[width=0.8\linewidth]{puntovista/dia-contexto.png}
	\caption{Diagrama de contexto.}
	\label{fig:dia-contexto}
\end{figure}

\subsection{Punto de Vista: Interacción}
El punto de vista de interacción tiene como propósito representar cómo los diferentes actores del sistema ChibchaWeb se comunican con los componentes funcionales durante la ejecución de casos de uso específicos.

\subsubsection{Diagrama de Casos de Uso}

Muestra la interacción entre los actores principales del sistema ChibchaWeb y los casos de uso que pueden ejecutar. Esta vista representa la funcionalidad del sistema desde el punto de vista de cada actor y su relación directa con los servicios ofrecidos, recalcar que aunque igual son numeros diagramas se intento ser lo más especifico en las funcionalidades planteadas para el proyecto, las funcionalidades de valor agregado al proyecto no fueron incluidas tales como buscar dominio por IA o el chatbot, el diagrama completo se encontrara en la sección de anexos.

\subsubsection{Diagramas de Secuencia}

Los diagramas de secuencia representan la dinámica del sistema durante la ejecución de casos de uso específicos. En ellos se detalla el orden cronológico de los mensajes intercambiados entre los actores y los objetos o componentes internos del sistema, todos estos diagramas se encontraran en la sección de anexos.\\

Esto Nos Permite Ver:

\begin{itemize}
\item Cómo se coordinan los distintos elementos para cumplir una funcionalidad.

\item Qué actores o servicios están involucrados.

\item Cómo fluye la lógica de negocio a través de las capas del sistema.
\end{itemize}


\subsubsection*{1. Agregar Método De Pago}

Este diagrama de secuencia representa el proceso mediante el cual un cliente registra un nuevo método de pago (tarjeta de crédito o débito) en la plataforma ChibchaWeb. El proceso inicia cuando el cliente ingresa los datos de su tarjeta en el formulario correspondiente.

El sistema separa internamente el flujo en dos pasos:

\begin{enumerate}
    \item Primero se realiza una llamada a la API /tarjeta para registrar los datos de la tarjeta en la base de datos.
    \item Luego, se ejecuta la llamada a /MetodoDePagopara asociar el método de pago a la cuenta del cliente.
\end{enumerate}

Ambas solicitudes son gestionadas por el módulo encargado de manejar los métodos de pago. Una vez completadas las operaciones en la base de datos, el sistema confirma al cliente que el método fue agregado correctamente.

\subsubsection*{2.  Registrarse Como Distribuidor}

Este diagrama de secuencia ilustra el proceso completo de registro de un nuevo distribuidor en la plataforma ChibchaWeb. El flujo inicia cuando el usuario llena el formulario de registro con su información personal y empresarial.

Una vez enviado el formulario:

\begin{enumerate}
\item El controlador de registros valida y procesa los datos recibidos.

\item Se genera una nueva cuenta en la base de datos del sistema.

\item Simultáneamente, el sistema genera un token de verificación y lo envía al correo electrónico del distribuidor mediante el servicio de correo.
\end{enumerate}

El distribuidor recibe un mensaje en pantalla indicando que debe verificar su cuenta para completar el proceso de registro y poder acceder al sistema con normalidad.

\subsubsection*{3. Registrarse Como Cliente}

Este diagrama de secuencia describe el proceso mediante el cual un cliente se registra en la plataforma ChibchaWeb. El flujo comienza cuando el cliente llena el formulario de registro desde el frontend.

El sistema ejecuta los siguientes pasos:

\begin{enumerate}
    \item La información del formulario es enviada a la API de registro.
    \item La API delega la validación de los datos a un servicio de validación.
    \item Una vez validados los campos, la API crea una nueva cuenta en la base de datos.
    \item Posteriormente, se genera un token de verificación a través del servicio de tokens, el cual es almacenado nuevamente en la base de datos.
    \item Finalmente, se utiliza el servicio de correo para enviar el mensaje de verificación al correo del cliente.
\end{enumerate}

El proceso concluye mostrando un mensaje al usuario con la instrucción de verificar su cuenta.

\subsubsection*{4. Generar archivo de dominio}

Este diagrama de secuencia representa el proceso completo mediante el cual un usuario envía una solicitud para registrar un dominio a través de la plataforma ChibchaWeb. El flujo incluye desde el ingreso inicial de los datos hasta la generación y envío del archivo XML al registrador externo.

El proceso sigue los siguientes pasos:

\begin{enumerate}
\item El usuario llena el formulario y envía la solicitud.
\item El controlador de dominio envía los datos al servicio de dominio, que los procesa y gestiona.
\item La solicitud se guarda a través del backend API, que la inserta en la base de datos y recibe un identificador.
\item Luego, se genera un archivo XML con los datos de la solicitud.
\item El archivo generado es enviado al registrador externo.
\item Tras la confirmación del registrador, se actualiza el estado de la solicitud en el sistema.
\item Finalmente, se muestra al usuario un mensaje confirmando que la solicitud fue enviada correctamente.
\end{enumerate}

\subsubsection*{5. Cargar Pagos a Tarjeta}
Este diagrama de secuencia muestra el proceso que se ejecuta cuando un cliente confirma la compra de un servicio desde el carrito, utilizando una tarjeta previamente registrada como método de pago.

El flujo se desarrolla de la siguiente forma:

 \begin{enumerate}
 \item El cliente confirma la compra desde el frontend, lo que activa el controlador de pagos.
 \item El controlador de pago solicita al backend API la verificación del método de pago seleccionado.
 \item El backend consulta la base de datos para validar la tarjeta asociada y, si es válida, procede a procesar la transacción con el servicio de pago externo.
 \item El servicio responde si el resultado fue aprobado o rechazado.
 \item En caso de aprobación:
 \begin{enumerate}
     \item Se confirma el pago del carrito.
     \item Se actualiza el estado del carrito en la base de datos.
     \item El cliente recibe un mensaje confirmando el éxito de la operación.
 \end{enumerate}

 \item Si el pago es rechazado:
    \begin{enumerate}
    	\item El sistema muestra un mensaje de error indicando que la transacción fue fallida.
    \end{enumerate}
 \end{enumerate}

\subsubsection*{6. Registrar Solicitud De Dominio}

Este diagrama de secuencia representa el flujo de interacción cuando un usuario desea registrar un dominio a través del sistema ChibchaWeb. La funcionalidad incluye una validación previa en tiempo real para comprobar si el nombre de dominio está disponible.

El proceso se detalla a continuación:

\begin{enumerate}
    \item El usuario ingresa el nombre del dominio en el formulario de registro.
    \item El controlador de registro de dominio consulta la disponibilidad del dominio utilizando un servicio de scraping, el cual accede a un registrador externo.
    \item Si el dominio está disponible:
    \begin{enumerate}
        \item Se realiza una solicitud POST para registrar el dominio a través del backend API.
        \item Esta solicitud se guarda en la base de datos, y se actualiza el estado del dominio como ocupado.
        \item El sistema muestra un mensaje indicando que la solicitud fue registrada exitosamente.
    \end{enumerate}

    \item Si el dominio no está disponible:
    \begin{enumerate}
    	\item Se muestra un mensaje de error al usuario indicando que el dominio ya está ocupado.
    \end{enumerate}
\end{enumerate}

\subsubsection*{7.Gestionar Perfiles De Clientes/Empleado}

Este diagrama de secuencia representa el proceso mediante el cual un distribuidor puede buscar, editar o eliminar perfiles de clientes en la plataforma.

El flujo incluye:

\begin{itemize}
    \item Búsqueda: El distribuidor consulta un cliente. El sistema recupera los datos desde la base de datos y los muestra en la interfaz.

    \item Edición: Se modifican los datos del perfil. La información se actualiza en la base de datos y se notifica al distribuidor.

    \item Eliminación: Se confirma la eliminación del perfil, la base de datos ejecuta la operación y se devuelve una respuesta de confirmación.
\end{itemize}

\subsubsection*{8. Calcular Comisión De Distribuidores}

Este diagrama de secuencia representa el proceso mediante el cual un administrador ejecuta el cálculo de comisiones para distribuidores en función de sus ventas registradas.

El flujo general es el siguiente:

\begin{enumerate}
    \item El administrador inicia el proceso desde la vista administrativa.
    \item El controlador de comisiones solicita al servicio de comisiones la ejecución del cálculo.
    \item Se consultan en la base de datos las ventas asociadas a cada distribuidor.
    \item El sistema aplica el porcentaje de comisión correspondiente ($10\% $ para distribuidores BÁSICOS y $15\%$ para PREMIUM).
\end{enumerate}

\subsubsection*{9. Generar Transferencia}

Este diagrama de secuencia describe el proceso mediante el cual un cliente transfiere la propiedad de un dominio a otro usuario de la plataforma, utilizando su dirección de correo como identificador.

El flujo se compone de los siguientes pasos:

\begin{enumerate}
    \item El cliente ingresa el dominio a transferir y el correo de destino en la interfaz correspondiente.
    \item El controlador de transferencia gestiona la solicitud y la envía al API de transferencia.
    \item El backend consulta la base de datos para verificar que la cuenta de destino existe y que el dominio pertenece al cliente actual.
    \item Si todo es válido, se actualiza el propietario del dominio en la base de datos.
    \item Finalmente, se muestra al cliente un mensaje indicando que la transferencia fue exitosa.
\end{enumerate}

\subsubsection*{10. Registrar Empleado}

 Este diagrama muestra el proceso mediante el cual un administrador registra un nuevo empleado en la plataforma ChibchaWeb.

El flujo sigue los siguientes pasos:

\begin{enumerate}
    \item El administrador completa el formulario de registro desde la vista administrativa.
    \item El controlador de registro de empleados recibe los datos y realiza una solicitud POST a la API de registro.
    \item La API almacena la información en la base de datos y devuelve una confirmación.
    \item Finalmente, el sistema muestra un mensaje notificando que el empleado fue registrado exitosamente.
\end{enumerate}

\subsubsection*{11. Registrar Tickets}

Este diagrama de secuencia representa el proceso mediante el cual un usuario crea un ticket de soporte en la plataforma ChibchaWeb, detallando un asunto y una descripción del problema.

El flujo es el siguiente:

\begin{enumerate}
    \item El usuario completa el formulario de soporte con los datos requeridos.
    \item El controlador de tickets procesa la solicitud y la envía mediante una petición POST a la API de creación de tickets.
    \item La API almacena el ticket en la base de datos y genera un identificador único.
    \item Finalmente, se muestra al usuario un mensaje de confirmación indicando que el ticket fue creado correctamente.
\end{enumerate}

\subsubsection*{12. Consultar Tickets}
Este diagrama de secuencia describe el proceso en el cual un usuario (cliente, distribuidor, empleado o soporte) accede al módulo de soporte técnico para consultar los tickets que le han sido asignados o que ha creado previamente.

El flujo contempla dos variantes según el tipo de usuario:

\begin{enumerate}
    \item Si el usuario es cliente o distribuidor, se realiza una solicitud GET /consultarTicketporIDCUENTA.
    \item Si es empleado o soporte técnico, se usa GET /mis-tickets-empleado.
\end{enumerate}

En ambos casos:

\begin{itemize}
    \item La solicitud es gestionada por el controlador, que la envía a la API de tickets.
    \item La API consulta la base de datos y devuelve la lista de tickets encontrados.
    \item Finalmente, la lista se muestra al usuario en la interfaz.
\end{itemize}

\subsubsection*{13. Atender Y Escalar Tickets}

Este diagrama de secuencia ilustra el proceso que realiza un usuario del área de soporte técnico para visualizar sus tickets asignados, actualizar su estado (por ejemplo, de "pendiente" a "en proceso" o "resuelto") y, si corresponde, escalar el ticket a un nivel superior.

El flujo incluye dos acciones principales:

\begin{enumerate}
\item \textbf{Atender Ticket (Cambiar Estado):}
\begin{itemize}
    \item El soporte accede al panel donde visualiza sus tickets.
    \item A través del controlador, el sistema obtiene los tickets desde la API y los recupera de la base de datos.
    \item El soporte selecciona un ticket y un nuevo estado, lo cual genera una solicitud PATCH /CambiarEstadoTicket.
    \item El estado se actualiza en la base de datos y se notifica al usuario.
\end{itemize}

\item \textbf{Escalar Ticket:}
\begin{itemize}
\item Si el ticket requiere mayor atención, el soporte ejecuta una acción para escalarlo.
\item Se envía una solicitud PATCH /CambiarNivelTicket con el identificador del ticket.
\item El backend registra el cambio de nivel, actualizando la información del ticket y notificando al usuario que el ticket fue escalado.
\end{itemize}
\end{enumerate}

\subsubsection*{14. Iniciar Sesión}

Este diagrama de secuencia describe el proceso de inicio de sesión en la plataforma ChibchaWeb por parte de un usuario, contemplando tres posibles resultados: acceso completo, acceso limitado (por cuenta no verificada) o error por credenciales inválidas.

El flujo es el siguiente:

\begin{enumerate}
\item El usuario ingresa sus credenciales en la vista de login.
\item El controlador de login las envía a la API, que consulta la base de datos para verificar:
\begin{itemize}
    \item La existencia del usuario.
    \item Que la contraseña coincida.
    \item Si la cuenta está verificada.
\end{itemize}

\item Si las credenciales son válidas:
\begin{enumerate}
    \item Si la cuenta está verificada: se inicia sesión, se devuelve un token y se redirige según el rol.
    \item Si no está verificada: se limita el acceso y se notifica al usuario que debe verificar su cuenta.
\end{enumerate}

\item Si las credenciales son inválidas:
\begin{enumerate}
	\item Se muestra un mensaje de error indicando el fallo.
\end{enumerate}
\end{enumerate}

\subsubsection*{15.Buscar Cuenta}

Este diagrama de secuencia muestra el proceso que sigue un administrador para consultar la información de una cuenta existente, utilizando como criterio de búsqueda un correo electrónico o ID.

El flujo es el siguiente:

\begin{enumerate}
    \item El administrador ingresa el correo o identificador en la vista de administración.
    \item El controlador de búsqueda de cuentas envía la solicitud al endpoint correspondiente de la API.
    \item La API consulta la base de datos para recuperar los datos de la cuenta asociada.
    \item Una vez encontrada, se retorna la información del perfil y se presenta al administrador.
\end{enumerate}

\subsubsection*{16. }

Este diagrama de secuencia describe el proceso mediante el cual un usuario accede a la sección “Mi Perfil” dentro de la plataforma ChibchaWeb para visualizar su información personal.

El flujo incluye los siguientes pasos:

\begin{enumerate}
\item El usuario navega a la vista de perfil desde la interfaz.
\item El controlador de perfil realiza una solicitud POST al endpoint /cuenta\_por\_correo en la API, pasando el identificador del usuario.
\item La API consulta la base de datos para obtener los datos asociados al correo.
\item Los datos del perfil (como nombre, correo, dirección, etc.) son devueltos y mostrados en pantalla.
\end{enumerate}

\subsubsection*{17. Validar Datos}

Este diagrama de secuencia describe el proceso de validación de datos ingresados por un usuario en formularios del sistema ChibchaWeb. Este flujo es común en procesos como registro, adición de tarjetas o dominios, y garantiza que la información ingresada cumpla con formato y disponibilidad antes de ser procesada.

El flujo contempla dos escenarios:

\begin{enumerate}
    \item \textbf{Datos válidos:}
    \begin{enumerate}
        \item El usuario ingresa los datos en el formulario.
        \item El frontend los envía al controlador de validación, que verifica el formato y la disponibilidad en la base de datos.
        \item Si no hay errores, se devuelve una respuesta confirmando que los datos fueron validados correctamente.
    \end{enumerate}

    \item \textbf{Datos inválidos o duplicados:}
    \begin{enumerate}
    	\item En caso de errores de formato o duplicidad (por ejemplo, correo o dominio ya registrados), el sistema responde con una lista de errores que se muestran al usuario.
    \end{enumerate}
\end{enumerate}

\subsection{Punto de Vista: Desarrollo}
En este diagrama, se aprecian tanto los componentes necesarios para el desarrollo del aplicativo, como su interacción con los demás, esto para poder tener una mayor comprensión del problema a solucionar.
\begin{figure}[H]
	\includegraphics[width=\columnwidth]{puntovista/diacomponentes.png}
	\caption{Diagrama de componentes}
	\label{fig:diacomponentes}
\end{figure}

\subsection{Punto de Vista: Modelo Relacional de la BBDD}
\begin{figure}[H]
    \centering
	\includegraphics[width=1\linewidth]{puntovista/dia-relacional.png}
	\caption{Diagrama de relacional.}
	\label{fig:dia-relacional}
\end{figure}


\section{Firmas de aceptación con fecha (DD/MM/AAAA)}

\newpage
\bibliographystyle{plain}
\bibliography{bibliografia/bibliografia}

\end{document}
