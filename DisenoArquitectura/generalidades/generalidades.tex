\subsection{Problema por resolver}
ChibchaWeb, una empresa de hospedaje de sitios web ubicada en Sugamuxi, ha experimentado un crecimiento constante en su base de clientes, principalmente en Colombia y países cercanos, y se proyecta una expansión hacia otros continentes como África. Actualmente administra miles de sitios web bajo distintos paquetes de hosting (Chibcha-Platino, Chibcha-Plata y Chibcha-Oro) sobre plataformas Windows y Unix. Los clientes pueden escoger planes de pago flexibles (mensual, trimestral, semestral o anual), y aunque el método principal de pago es con tarjeta de crédito, se planea habilitar próximamente otros medios como PSE.

El modelo de negocio incluye además una red de distribuidores categorizados como BÁSICOS o PREMIUM, con diferentes esquemas de comisiones. ChibchaWeb también intermedia en el registro y transferencia de dominios a través de terceros registradores, ofreciendo este servicio adicional a sus clientes. A medida que el volumen de operaciones crece y los procesos se diversifican, la compañía enfrenta varios desafíos operativos y administrativos.

Actualmente, ChibchaWeb carece de una solución unificada que automatice y centralice sus procesos comerciales, operativos y de atención al cliente. Esto genera ineficiencias en la gestión de perfiles, pagos, comisiones, atención técnica y comunicación con registradores externos. La falta de un sistema integral limita la capacidad de escalar operaciones, introducir nuevos métodos de pago, y garantizar trazabilidad y control sobre los procesos internos.

\subsection{Descripción general del sistema a desarrollar}
El sistema a desarrollar para ChibchaWeb es una plataforma integral de gestión de servicios de hospedaje web, diseñada para automatizar, centralizar y optimizar los procesos operativos clave de la compañía. Esta solución tecnológica permitirá gestionar clientes, empleados, distribuidores, dominios, pagos y soporte técnico de manera eficiente, segura y escalable.

El sistema contará con módulos para la creación y administración de perfiles de clientes y empleados, y almacenará la información en una base de datos relacional optimizada incluyendo la validación de información personal y financiera garantizando la integridad evitando ciclos en esta; buscar y adquirir dominios mediante un carrito; adquirir paquetes de hosting.\\

El sistema contará con roles diferenciados por tipo de cuenta; búsquedas y consultas detalladas de todos los actores del sistema, así como herramientas para la generación de archivos de transferencia electrónica y seguimiento de actividades internas del proyecto; gestión de solicitudes de dominios con generación de archivos XML para su envío a registradores externos; Cálculo automático de comisiones a distribuidores según su categoría (BÁSICO o PREMIUM); y un módulo de soporte técnico basado en tickets que garantiza el seguimiento y resolución estructurada de problemas reportados.\\

La solución incluirá una interfaz web en React, backend en FastAPI y base de datos en MySQL. Se aplicarán buenas prácticas de diseño para mantener una l ógica de negocio clara y una estructura coherente en el modelo de datos.

\subsection{Objetivos de la solución}

\subsubsection{Objetivo general}
\begin{itemize}
	\item Desarrollar un sistema web que permita registrar, gestionar y monitorear cuentas de usuarios, adquisición de dominios y facturación de manera eficiente y centralizada.

\end{itemize}

\subsubsection{Objetivos específicos}
\begin{itemize}
\item Diseñar una base de datos relacional normalizada y sin ciclos innecesarios para soportar la lógica del sistema.
\item Desarrollar una interfaz web funcional y responsiva utilizando React y Bootstrap para facilitar la interacción del usuario con el sistema.
\item Construir un backend en FastAPI que gestione la lógica del negocio, la autenticación de usuarios y la comunicación con la base de datos.
\end{itemize}

\subsection{Stakeholders}
%configutación tablas
\pgfplotstableset{
    % Estilo general
    every head row/.style={
        before row=\hline\rowcolor{cafet}\bfseries,
        after row=\hline,
    },
    every odd row/.style={before row=\rowcolor{gray!10}},
    columns/Stakeholder/.style={string type, column type=p{4cm}},
    columns/Descripción/.style={string type,column type=p{12cm},},
    col sep=semicolon,
}

\begin{center}
    \begin{table}[H]
    \pgfplotstabletypeset[
    col sep=semicolon,
    header=true,
    ]{generalidades/stakeholders.dat}
    \caption{Stakeholders}
    \end{table}
\end{center}
