\subsubsection*{A}
\begin{itemize}
    \item \textbf{Administrador del sistema:} Usuario con permisos elevados para gestionar configuraciones, cuentas, dominios y operaciones críticas del sistema.
\end{itemize}

\subsubsection*{B}

\subsubsection*{C}
\begin{itemize}
\item \textbf{Carrito de compra:} Funcionalidad que permite a los usuarios seleccionar, almacenar y gestionar productos o servicios antes de completar una compra.

\item \textbf{Cliente (cuenta):} Usuario registrado en el sistema que adquiere y gestiona servicios como dominios y hosting.

\item \textbf{Comisión:} Porcentaje o monto fijo asignado a distribuidores por ventas realizadas dentro del sistema.

\item \textbf{Cuenta (CUENTA):} Registro individual de usuario dentro del sistema, ya sea cliente, distribuidor, empleado o administrador.
\end{itemize}

\subsubsection*{D}
\begin{itemize}
    \item \textbf{Distribuidor:} Usuario con permisos especiales para revender servicios de dominio/hosting, gestionando clientes y comisiones.
    \item \textbf{Dominio (DOMINIO):} Nombre único que identifica un sitio en Internet, adquirido y gestionado por usuarios dentro del sistema.
    \item \textbf{Documentación técnica:} Conjunto de manuales, diagramas y guías destinados a facilitar el desarrollo, uso y mantenimiento del sistema.
\end{itemize}

\subsubsection*{E}
\begin{itemize}
    \item \textbf{Estado del carrito:} Condición actual del carrito de un usuario (abierto, confirmado, pagado, cancelado, etc.).
    \item \textbf{Estado del ticket:} Seguimiento del progreso de un ticket de soporte (pendiente, en proceso, resuelto, cerrado).
    \item \textbf{Empleado (cuenta):} Usuario con rol operativo dentro del sistema, encargado de soporte o administración técnica.
\end{itemize}

\subsubsection*{F}

\subsubsection*{G}
\begin{itemize}
    \item \textbf{Gestión de tickets:} Funcionalidad que permite registrar, asignar y resolver solicitudes o incidencias de usuarios.
    \item \textbf{Gestión de dominios:} Módulo que permite registrar, transferir o administrar dominios vinculados a los clientes.
    \item \textbf{Gestión de cuentas:} Funcionalidad administrativa para crear, editar, suspender o eliminar cuentas del sistema.
\end{itemize}

\subsubsection*{H}
\begin{itemize}
\item \textbf {Hosting:} Servicio de alojamiento web que permite publicar sitios y aplicaciones en Internet, con diferentes planes disponibles.
\end{itemize}

\subsubsection*{I}
\begin{itemize}
    \item \textbf {Identificación de cuenta:} Proceso que permite determinar el tipo, estado y permisos de una cuenta específica.
    \item \textbf {ItemPaquete:} Unidad o ítem que forma parte de un paquete de hosting, dominio u otro servicio agregado al carrito.
\end{itemize}

\subsubsection*{J}

\subsubsection*{K}

\subsubsection*{L}

\subsubsection*{M}
\begin{itemize}
    \item \textbf {Método de pago:} Forma en que un cliente puede abonar por servicios (tarjeta, transferencia, etc.).
\end{itemize}

\subsubsection*{N}

\subsubsection*{O}

\subsubsection*{P}
\begin{itemize}
    \item \textbf{Paquete de hosting:} Conjunto de servicios agrupados bajo un plan que incluye almacenamiento, ancho de banda, correos, etc.
    \item \textbf{Plan (de hosting):} Nivel de servicio contratado, con características técnicas y precio definido.
\end{itemize}

\subsubsection*{Q}

\subsubsection*{R}
\begin{itemize}
    \item \textbf{Registro de cuenta:} Proceso por el cual un nuevo usuario se da de alta en el sistema.
    \item \textbf{Roles de usuario:} Clasificación de cuentas según sus privilegios (cliente, empleado, administrador, distribuidor, etc.).
\end{itemize}

\subsubsection*{S}
\begin{itemize}
    \item \textbf{Segmentación de cuentas:} Clasificación y asignación de cuentas según criterios administrativos o funcionales.
    \item \textbf{Soporte técnico:} Área dedicada a resolver consultas, problemas o solicitudes de los usuarios del sistema.
    \item \textbf{Sistema de autenticación:} Mecanismo que valida la identidad de usuarios y otorga acceso según sus permisos.
\end{itemize}

\subsubsection*{T}
\begin{itemize}
\item \textbf{Tarjeta (de pago):} Medio electrónico utilizado por los clientes para abonar productos o servicios.
\item \textbf{Ticket:} Solicitud registrada por un cliente relacionada con soporte técnico, facturación, etc.
\item \textbf{Token:} Cadena generada para identificar sesiones, validar acciones o proteger datos en tránsito.
\item \textbf{Transferencia de dominio:} Proceso mediante el cual un dominio cambia de registrador o de cuenta dentro del sistema.
\item \textbf{Tipo de cuenta:} Clasificación de una cuenta según su función: cliente, empleado, distribuidor, etc.
\item \textbf{Tipo de método de pago:} Categoría asignada al método usado para abonar: tarjeta, efectivo, transferencia.
\end{itemize}

\subsubsection*{U}
\begin{itemize}
\item \textbf{Usuario final:} Persona que utiliza el sistema para contratar servicios y administrar su cuenta.
\end{itemize}

\subsubsection*{V}

\subsubsection*{W}

\subsubsection*{X}

\subsubsection*{Y}

\subsubsection*{Z}
